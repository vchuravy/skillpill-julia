%%%%%%%%%%%%%%%%%%%%%%%%%%%%%%%%%%%%%%%%%
% Beamer Presentation
% LaTeX Template
% Version 1.0 (10/11/12)
%
% This template has been downloaded from:
% http://www.LaTeXTemplates.com
%
% License:
% CC BY-NC-SA 3.0 (http://creativecommons.org/licenses/by-nc-sa/3.0/)
%
% Modified by Jeremie Gillet in November 2015 to make an OIST Skill Pill template
%
%%%%%%%%%%%%%%%%%%%%%%%%%%%%%%%%%%%%%%%%%

%----------------------------------------------------------------------------------------
%	PACKAGES AND THEMES
%----------------------------------------------------------------------------------------

\documentclass{beamer}

\mode<presentation> {

\usetheme{Madrid}

\definecolor{OISTcolor}{rgb}{0.65,0.16,0.16}
\usecolortheme[named=OISTcolor]{structure}

%\setbeamertemplate{footline} % To remove the footer line in all slides uncomment this line
%\setbeamertemplate{footline}[page number] % To replace the footer line in all slides with a simple slide count uncomment this line

\setbeamertemplate{navigation symbols}{} % To remove the navigation symbols from the bottom of all slides uncomment this line
}

\usepackage{graphicx} % Allows including images
\usepackage{booktabs} % Allows the use of \toprule, \midrule and \bottomrule in tables
\usepackage{textpos} % Use for positioning the Skill Pill logo
\usepackage{fancyvrb}
\usepackage{tikz}
\usepackage{hyperref}
\usepackage{listings}

\definecolor{dkgreen}{rgb}{0,0.6,0}
\definecolor{gray}{rgb}{0.5,0.5,0.5}
\definecolor{mauve}{rgb}{0.58,0,0.82}

\lstset{frame=tb,
  language=python,
  aboveskip=3mm,
  belowskip=3mm,
  showstringspaces=false,
  columns=flexible,
  basicstyle={\small\ttfamily},
  numbers=none,
  numberstyle=\tiny\color{gray},
  keywordstyle=\color{blue},
  commentstyle=\color{dkgreen},
  stringstyle=\color{mauve},
  breaklines=true,
  breakatwhitespace=true,
  tabsize=3
}

%----------------------------------------------------------------------------------------
%	TITLE PAGE
%----------------------------------------------------------------------------------------

\title[Skill Pill]{Skill Pill: Julia} % The short title appears at the bottom of every slide, the full title is only on the title page
\subtitle{Lecture 1: Introduction}

\author{James Schloss \and Valentin Churavy} % Your name
\institute[OIST] % Your institution as it will appear on the bottom of every slide, may be shorthand to save space
{
Okinawa Institute of Science and Technology \\ % Your institution for the title page
\textit{james.schloss@oist.jp}\\
\textit{valentin.churavy@oist.jp} % Your email address
}
\date{July 7, 2017} % Date, can be changed to a custom date

\begin{document}

\setbeamertemplate{background}{\includegraphics[width=\paperwidth]{SPbackground.png}} % Adding the background logo

\begin{frame}
\vspace*{1.4cm}
\titlepage % Print the title page as the first slide
\end{frame}

\setbeamertemplate{background}{} % No background logo after title frame

\addtobeamertemplate{frametitle}{}{% Adding the Skill Pill logo on the title screen after title frame
\begin{textblock*}{100mm}(.92\textwidth,-0.9cm)
\includegraphics[height=0.85cm]{julia.pdf}
\end{textblock*}}

\begin{frame}
  \tableofcontents
\end{frame}
\section{The foreign world, using Julia to reuse prior work}
\begin{frame}[fragile]{Using Fortran and C in Julia}

Julia allows you to use other languages (such as Fortran or C) by using the \texttt{ccall} function:

\begin{lstlisting}
    julia> t = ccall((:clock, "libc"), Int32, ())
    2292761
\end{lstlisting}

Here, we are calling the \texttt{clock} function from the \texttt{libc} library in C.

\end{frame}

\begin{frame}[fragile]{Your legacy code}
Let's say you want to use a simply multiply function in Fortran:
\begin{lstlisting}[language=fortran]
      !! We'll be using subroutines intead of functions 
      subroutine multiply(A, B, C)
          REAL*8 :: A, B, C
          C = A * B
          return
      end
\end{lstlisting}
\vspace{0.5cm}
or C:
\begin{lstlisting}[language=c]
    // Nothing fancy here...
    double multiply(double A, double B){
        return A*B;
    }
\end{lstlisting}
\end{frame}

\begin{frame}[fragile]{Preparing your legacy code}

In order to use your favorite C or Fortran code in Julia, you need to compile it into a library, like so:

\begin{lstlisting}
    gcc -shared -O2 multiply.c -fPIC -o c_multiply.so
    gfortran -shared -O2 multiply.f90 -fPIC -o 
        fortran_multiply.so
\end{lstlisting}

\pause
These will create libraries with all of the necessary functions you could want, but beware:

\begin{center}
\textbf{C and Fortran compilers mangle function names!}
\end{center}
\end{frame}

\begin{frame}[fragile]{Using your legacy code}
There are 3 things to keep. Make sure you 
\begin{enumerate}
\item Have the right mangled name
\item Are using the right type
\item Are using the function correctly.
\end{enumerate}
\pause

\vspace{0.5cm}
For example, in C:
\begin{lstlisting}
    # This function multiplies a and b into c by using the 
    # created C library
    function call_c()
        a = Cdouble(1.0)
        b = Cdouble(3.0)
        c = ccall((:multiply, "/full/path/to/c_multiply"),
            Cdouble,(Cdouble, Cdouble),a,b)
        println(c)
    end
\end{lstlisting}
\end{frame}

\begin{frame}[fragile]{Using your legacy code}
Pointers are okay! For example, in Fortran:

\begin{lstlisting}
    # This function multiplies a and b into c by using 
    # the created FORTRAN library
    function call_fortran()
        a = Cdouble[1.0]
        b = Cdouble[2.0]
        c = Cdouble[0.0]
        ccall((:multiply_, "/full/path/to/fortran_multiply"),
           Void,(Ptr{Float64},Ptr{Float64},Ptr{Float64}),
                 a,b,c)
        println(c[1])
    end
\end{lstlisting}

\pause
More information can be found here: \url{https://docs.julialang.org/en/stable/manual/calling-c-and-fortran-code/}
\end{frame}
\begin{frame}{Support for other languages}
  \begin{description}
    \item[Python] https://github.com/JuliaPy/PyCall.jl
    \item[R] https://github.com/JuliaInterop/RCall.jl
    \item[C++] https://github.com/Keno/Cxx.jl
    \item[Matlab] I have heard rumours of such a thing existing, but the horror $\cdots$
  \end{description}
  \begin{block}{Conclusion}
    Start writing Julia code now without being worried about losing your prior work!
  \end{block}
\end{frame}

\section{Macros and metaprogramming}
\begin{frame}{Macros and metaprogramming}
\end{frame}
\section{The Julia compiler}
\begin{frame}[fragile]{The stages of the compiler}
  \begin{enumerate}
    \item Surface syntax (the code you write)
    \item Desugared AST \lstinline{@code\_lowered}
    \item Type-inferred AST \lstinline{@code\_typed}
    \item LLVM IR \lstinline{@code\_llvm}
    \item Native assembly \lstinline{@code\_native}
  \end{enumerate}
\end{frame}
\section{Performance}
\begin{frame}[fragile]{Methodology}
  \pause
  \begin{block}{Measure first}
    Before you start iterating on your code establish a baseline performance.
    Computers are noisy system so we use the lowest runtime as a metric.
  \end{block}
  \pause
  \begin{itemize}
    \item Check for type-instabilities with \lstinline{@code\_warntype}
    \item Measure runtime and allocations with \lstinline{@time}
    \item Benchmark using \lstinline{@btime}, and \lstinline{@benchmark} from \lstinline{BenchmarkTools.jl}
    \item Profiler and \lstinline{ProfileView.jl}
    \item Memory Allocation tracker
  \end{itemize}
  \pause
  Read the performance tips section of the Julia manual \url{https://docs.julialang.org/en/stable/manual/performance-tips/}
\end{frame}
\begin{frame}[fragile]{Type instabilities}
\end{frame}
\begin{frame}[fragile]{Using BenchmarksTools.jl}
\end{frame}
\begin{frame}[fragile]{Using the Profiler}
  You can profile a piece of code with Julia's inbuilt profiler.
  \begin{description}
    \item[\lstinline{@profile fun()}] Profile a specific function
    \item[\lstinline{Profile.clear()}] Clear the recorded profile
    \item[\lstinline{Profile.print()}] Print the profile
    \item[\lstinline{Profile.print(C=true)}] Print the profile including stacktraces reaching into C.
  \end{description}
  \begin{block}{ProfileView.jl}
    The textual output of the profiler can be hard understand \lstinline{ProfileView.jl} gives a graphical representation.
    \begin{lstlisting}
    using ProfileView
    ProfileView.view()
    \end{lstlisting}
  \end{block}
\end{frame}
\begin{frame}[fragile]{Using the memory allocation tracker}
  To track memory allocations you have to start Julia with the memory allocation tracker enabled.
  \begin{lstlisting}
  # Track only allocation in user code
  julia --track-allocation=user
  # Track allocation in all code (includeing the Julia base)
  julia --track-allocation=all
  \end{lstlisting}
  After quiting Julia *.mem files are created that contain cumulative amounts of allocated memory.
  \begin{block}{Getting useful data}
    Since we have to start Julia with track allocations enable we will gather a lot of noisy data.
    To cut down the noise run your code in a session once and then use \lstinline{Profile.clear\_malloc\_data()}
    to reset the allocation counts and then run your code again only tracking revelant allocations.
  \end{block}
\end{frame}
\begin{frame}[fragile]{A simple example}
\begin{lstlisting}
function mysum(A)
  acc = 0
  for x in A
     acc += x
  end
  return acc
end
\end{lstlisting}
\end{frame}
\section{Performance analysis}
\begin{frame}[fragile]{A supposedly simple task}
  \begin{lstlisting}
  function myfun()
    s = 0.0
    N = 10000
    for i=1:N
        s+=det(randn(3,3))
    end
    s/N
  end
  \end{lstlisting}
\end{frame}

\begin{frame}{What is next?}
  \begin{description}
    \item[Next Session] Data Structures and Algorithms
    \item[Last Session] Parallel computing, threading, GPUs? Up to grabs.
  \end{description}
\end{frame}
\end{document}
