%%%%%%%%%%%%%%%%%%%%%%%%%%%%%%%%%%%%%%%%%
% Beamer Presentation
% LaTeX Template
% Version 1.0 (10/11/12)
%
% This template has been downloaded from:
% http://www.LaTeXTemplates.com
%
% License:
% CC BY-NC-SA 3.0 (http://creativecommons.org/licenses/by-nc-sa/3.0/)
%
% Modified by Jeremie Gillet in November 2015 to make an OIST Skill Pill template
%
%%%%%%%%%%%%%%%%%%%%%%%%%%%%%%%%%%%%%%%%%

%----------------------------------------------------------------------------------------
%	PACKAGES AND THEMES
%----------------------------------------------------------------------------------------

\documentclass{beamer}

\mode<presentation> {

\usetheme{Madrid}

\definecolor{OISTcolor}{rgb}{0.65,0.16,0.16}
\usecolortheme[named=OISTcolor]{structure}

%\setbeamertemplate{footline} % To remove the footer line in all slides uncomment this line
%\setbeamertemplate{footline}[page number] % To replace the footer line in all slides with a simple slide count uncomment this line

\setbeamertemplate{navigation symbols}{} % To remove the navigation symbols from the bottom of all slides uncomment this line
}
\setbeamertemplate{itemize items}[default]
\setbeamertemplate{enumerate items}[default]
\usepackage{graphicx} % Allows including images
\usepackage{booktabs} % Allows the use of \toprule, \midrule and \bottomrule in tables
\usepackage{textpos} % Use for positioning the Skill Pill logo
\usepackage{fancyvrb}
\usepackage{tikz}
\usepackage{hyperref}
\usepackage{listings}
\usepackage{color}

\definecolor{dkgreen}{rgb}{0,0.6,0}
\definecolor{gray}{rgb}{0.5,0.5,0.5}
\definecolor{mauve}{rgb}{0.58,0,0.82}

\lstset{frame=tb,
  language=python,
  aboveskip=3mm,
  belowskip=3mm,
  showstringspaces=false,
  columns=flexible,
  basicstyle={\small\ttfamily},
  numbers=none,
  numberstyle=\tiny\color{gray},
  keywordstyle=\color{blue},
  commentstyle=\color{dkgreen},
  stringstyle=\color{mauve},
  breaklines=true,
  breakatwhitespace=true,
  tabsize=3
}

\newcommand*{\lstitem}[1]{
  \setbox0\hbox{\lstinline{#1}}  
  \item[\usebox0]  
  % \item[\hbox{\lstinline{#1}}]
  \hfill \\
}

%----------------------------------------------------------------------------------------
%	TITLE PAGE
%----------------------------------------------------------------------------------------

\title[Skill Pill]{Skill Pill: Julia} % The short title appears at the bottom of every slide, the full title is only on the title page
\subtitle{Lecture 3: Data Structures and Algorithms}

\author{Valentin Churavy \and James Schloss} % Your name
\institute[OIST] % Your institution as it will appear on the bottom of every slide, may be shorthand to save space
{
Okinawa Institute of Science and Technology \\ % Your institution for the title page
\textit{valentin.churavy@oist.jp} \\
\textit{james.schloss@oist.jp} % Your email address
}
\date{July 7, 2017} % Date, can be changed to a custom date

\begin{document}

\setbeamertemplate{background}{\includegraphics[width=\paperwidth]{SPbackground.png}} % Adding the background logo

\begin{frame}
\vspace*{1.4cm}
\titlepage % Print the title page as the first slide
\end{frame}

\setbeamertemplate{background}{} % No background logo after title frame

\addtobeamertemplate{frametitle}{}{% Adding the Skill Pill logo on the title screen after title frame
\begin{textblock*}{100mm}(.92\textwidth,-0.9cm)
\includegraphics[height=0.85cm]{julia.pdf}
\end{textblock*}}

\begin{frame}
  \tableofcontents
\end{frame}

\begin{frame}
\frametitle{Disclaimer}

When designing this Skill Pill, we assumed the following
\begin{itemize}
\item You have seen and are familiar with common data structures
\item You know how to program and use programming as part of your daily work. 
\end{itemize}

As such, we have designed today's lesson so that you may begin using Julia in your work as soon as possible.

\end{frame}
\section{Data Structures}

\begin{frame}
\frametitle{Datastructures.jl}

\begin{columns}
\column{0.5\textwidth}
DataStructures.jl has the following data structures:
\begin{itemize}
\item Deque (based on block-list)
\item Stacks and Queues
\item Accumulators and Counters
\item Disjoint Sets
\item Binary Heap
\item Mutable Binary Heap
\item Ordered Dicts and Sets
\item Dictionaries with Defaults
\item Trie (Tree)
\item Linked List
\item Sorted Dict, Multi-Dict and Set
\item Priority Queue
\end{itemize}
\pause
\column{0.5\textwidth}
All information regarding these Data Structures can be found Here: \url{ http://datastructuresjl.readthedocs.io/en/latest/index.html/}

\vspace{0.5cm}
All of these algorithms can be viewed with \texttt{@edit}. We'll use two more before hopping into algorithms: \texttt{Binary Trees} and \texttt{Priority Queues}
\end{columns}

\end{frame}


\begin{frame}[fragile]
\frametitle{Using simple trees}
First, let's get used to using DataStructures.jl by Depth-First searching in binary trees. You might be used to binary 	tree nodes that look like this (C++):
\begin{lstlisting}[language=c++]
struct node{
    double weight;

    node *left;
    node *right;
    node *parent;
};
\end{lstlisting}

Basically, each node has parents and children.
\end{frame}

\begin{frame}[fragile]
\frametitle{Using simple trees}
In Julia, the tree nodes might look like:
\begin{lstlisting}
typealias BT{T} Union{T,Empty}

type BTree
    weight::Float64
    left::BT{BTree}
    right::BT{BTree}
end

\end{lstlisting}
\end{frame}

\begin{frame}[fragile]
\frametitle{Depth-First-Search}
To search through a tree, we need a \textbf{recursive} strategy. One of these strategies is known as \textbf{Depth-First Search} (or corresponding \textbf{Breadth-First Search}). 

\vspace{0.5cm}
These searching algorithms always go from the root $\rightarrow$ nodes. In C++:
\begin{lstlisting}[language=c++]
void depth_first_search(node* &node){
    if (root->right){
        depth_first_search(node->right);
    }
    if (root->left){
        depth_first_search(node->left);
    }
}
\end{lstlisting}
\end{frame}

\begin{frame}[fragile]
\frametitle{Depth-First Search}
In Julia, this looks like:
\begin{lstlisting}
function DFS(node::BTree)
    if (node.right != et)
        DFS(node.right)
    end
    
    if (node.left) != et
        DFS(node.left)
    end
end

\end{lstlisting}
\begin{block}{Exercise}
Write a DFS that acts on a binary tree and output a binary string that traverses to a leaf node.
\end{block}
\end{frame}

\begin{frame}[fragile]
\frametitle{PriorityQueue}
\texttt{Priority Queues} are found in the \texttt{DataStructures.jl} Package:
\begin{lstlisting}
julia> using DataStructures

julia> pq = PriorityQueue()

julia> pq["a"] = 15
julia> pq["b"] = 20

julia> pq
DataStructures.PriorityQueue{Any,Any,Base.Order.ForwardOrdering} with 2 entries:
  "b" => 20
  "a" => 15

julia> dequeue!(pq)
"a"
\end{lstlisting}
\end{frame}

\begin{frame}
\frametitle{Other Data Structures}
Obviously, all your favorite data structures can be implemented in Julia, but for now, we will move on to...
\end{frame}

\section{Algorithms}
\begin{frame}
\frametitle{Algorithms in Julia}

\pause
\begin{center}
\huge{Implement your favorite algorithm in Julia}

\vspace{0.5cm}
\small PS: We're here to help!
\end{center}

\end{frame}

\begin{frame}{What is next?}
  \begin{description}
    \item[Last Session] Parallel computing, threading, GPUs? Up to grabs.
  \end{description}

Join us for the exciting conclusion!
\end{frame}
\end{document}
