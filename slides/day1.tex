%%%%%%%%%%%%%%%%%%%%%%%%%%%%%%%%%%%%%%%%%
% Beamer Presentation
% LaTeX Template
% Version 1.0 (10/11/12)
%
% This template has been downloaded from:
% http://www.LaTeXTemplates.com
%
% License:
% CC BY-NC-SA 3.0 (http://creativecommons.org/licenses/by-nc-sa/3.0/)
%
% Modified by Jeremie Gillet in November 2015 to make an OIST Skill Pill template
%
%%%%%%%%%%%%%%%%%%%%%%%%%%%%%%%%%%%%%%%%%

%----------------------------------------------------------------------------------------
%	PACKAGES AND THEMES
%----------------------------------------------------------------------------------------

\documentclass{beamer}

\mode<presentation> {

\usetheme{Madrid}

\definecolor{OISTcolor}{rgb}{0.65,0.16,0.16}
\usecolortheme[named=OISTcolor]{structure}

%\setbeamertemplate{footline} % To remove the footer line in all slides uncomment this line
%\setbeamertemplate{footline}[page number] % To replace the footer line in all slides with a simple slide count uncomment this line

\setbeamertemplate{navigation symbols}{} % To remove the navigation symbols from the bottom of all slides uncomment this line
}

\usepackage{graphicx} % Allows including images
\usepackage{booktabs} % Allows the use of \toprule, \midrule and \bottomrule in tables
\usepackage{textpos} % Use for positioning the Skill Pill logo
\usepackage{fancyvrb}
\usepackage{tikz}
\usepackage{hyperref}
\usepackage{listings}


%----------------------------------------------------------------------------------------
%	TITLE PAGE
%----------------------------------------------------------------------------------------

\title[Skill Pill]{Skill Pill: Julia} % The short title appears at the bottom of every slide, the full title is only on the title page
\subtitle{Lecture 1: Introduction}

\author{James Schloss \and Valentin Churavy} % Your name
\institute[OIST] % Your institution as it will appear on the bottom of every slide, may be shorthand to save space
{
Okinawa Institute of Science and Technology \\ % Your institution for the title page
\textit{james.schloss@oist.jp}\\
\textit{valentin.churavy@oist.jp} % Your email address
}
\date{July 4, 2017} % Date, can be changed to a custom date

\begin{document}

\setbeamertemplate{background}{\includegraphics[width=\paperwidth]{SPbackground.png}} % Adding the background logo

\begin{frame}
\vspace*{1.4cm}
\titlepage % Print the title page as the first slide
\end{frame}

\setbeamertemplate{background}{} % No background logo after title frame

\addtobeamertemplate{frametitle}{}{% Adding the Skill Pill logo on the title screen after title frame
\begin{textblock*}{100mm}(.92\textwidth,-0.9cm)
\includegraphics[height=0.85cm]{julia.pdf}
\end{textblock*}}

\section{Introduction}
\begin{frame}{A short history of Julia}
  \begin{description}
    \item[dawn of time] 
    \item[0.1]
    \item[0.2]
    \item[0.3]
    \item[0.4]
    \item[0.5]
  \end{description}
\end{frame}

\begin{frame}{Installation}
  \begin{block}{Windows, Linux, and Mac OSX}
    Download a precompiled version of 0.6 from \url{https://julialang.org/downloads/}
  \end{block}
  \begin{block}{Linux and Mac OSX}
    \begin{enumerate}
      \item Use your package manager (Mac OSX: \texttt{brew cask install julia})
      \item Follow the build instructions from \url{https://github.com/JuliaLang/julia/}
    \end{enumerate}
  \end{block}
  \begin{block}{Sango and Tombo}
    OIST has Julia installed on Sango and Tombo, in case the version you need is not there let \href{mailto:it-help@oist.jp}{it-help@oist.jp} know.
  \end{block}
\end{frame}

\begin{frame}{Resources}
  \begin{description}
    \item[Documentation] \url{https://docs.julialang.org/en/release-0.6/}
    \item[Forum] \url{https://discourse.julialang.org}
    \item[Issue Tracker] \url{https://github.com/JuliaLang/julia}
    \item[Downloads] \url{https://julialang.org/downloads/}
    \item[Packages] \url{https://pkg.julialang.org/}
    \item \url{https://juliaobserver.com/}
  \end{description}
\end{frame}

\subsection{Why does Julia exist}
\begin{frame}{Why does Julia exist?}
  Statement: Scientist like high-level programming languages
  - Why Julia thread on discourse
  - old blogposts
\end{frame}
\subsection{What is wrong with the status quo}
\begin{frame}{The other contenders}
  The typical languages used in science are
  \begin{enumerate}
    \item Python
    \item Matlab
    \item R
  \end{enumerate}

  Once a problem is becoming to big we usually move to
  \begin{enumerate}
    \item C/C++
    \item Fortran
  \end{enumerate}
  This is called the 2+ language problem and Julia is trying to solve that.
\end{frame}

\begin{frame}{Performance vs Productivity}
  - Find that slide
\end{frame}

\begin{frame}{Python and Numpy}
  Hard to optimize, JIT limited, GIL, fast code needs to be written in C
  Numpy gets in the way when writting scientific code
\end{frame}

\begin{frame}{Matlab}
  Only fast for the subst of operation mathworks deemed important
  Cost
\end{frame}

\begin{frame}{R}
  Slow, don't even try to do numerics
\end{frame}

\begin{frame}{A (biased) performance comparision}
  Find slide
\end{frame}

\section{Getting started}
\begin{frame}[fragile]{The REPL}
  \begin{block}{The Read-Eval-Print-Loop}
    The REPL is a command-line interface to Julia and is ideal for short experiments.
    \begin{Verbatim}[fontsize=\footnotesize]
               _
   _       _ _(_)_     |  A fresh approach to technical computing
  (_)     | (_) (_)    |  Documentation: https://docs.julialang.org
   _ _   _| |_  __ _   |  Type "?help" for help.
  | | | | | | |/ _` |  |
  | | |_| | | | (_| |  |  Version 0.6.0 (2017-06-19 13:05 UTC)
 _/ |\__'_|_|_|\__'_|  |  
|__/                   |  x86_64-pc-linux-gnu

julia> 

\end{Verbatim}
  \end{block}
\end{frame}
\begin{frame}{IDEs}
  Juno/Atom
  VStudioCode
\end{frame}
\begin{frame}[fragile]{Jupyter}
  Jupyter is an interactive web-based client for Python, Julia, R and many other languages.
  It offers a programming environment that is well suited for explorative data analysis or prototyping.
  \begin{block}{Installation}
  \begin{Verbatim}
    julia> ENV["JUPYTER"] = ""
    julia> Pkg.add("IJulia")
    \end{Verbatim}
  \end{block}
  \begin{block}{Starting a Jupyter session}
    \begin{Verbatim}
    julia> using IJulia
    julia> notebook()
    \end{Verbatim}
  \end{block}
  \begin{block}{JuliaBox}
    There is an online service provided by JuliaComputing at \url{https://juliabox.com} that gives you a cloud version of Jupyter.
  \end{block}
\end{frame}
\section{Language syntax and semantics}
\begin{frame}[fragile]{Variables and datatypes}
  Julia is a dynamic language and so you can simply create variables in any scope.
  \begin{Verbatim}
  x = 1   # x will be of type Int64
  y = 1.0 # y will be of type Float64
  z = 1.0 - 2.0im # z will be an Complex{Float64}
  1//2
  ""
  ''
  6.1e6

  \end{Verbatim}
\end{frame}
\begin{frame}[fragile]{Conditionals}
\end{frame}
\begin{frame}[fragile]{Loops}
\end{frame}
\begin{frame}[fragile]{Functions and lambdas}
\end{frame}
\begin{frame}[fragile]{Types}
\end{frame}
\begin{frame}[fragile]{Multiple dispatch in a nutshell}
\end{frame}
\begin{frame}[fragile]{Modules}
\end{frame}
\section{The ecosystem}
\begin{frame}{Installing packages}
\end{frame}
\begin{frame}{Github}
\end{frame}
\section{The foreign world, using Julia to reuse prior work}

\begin{frame}[fragile]{Using Fortran and C in Julia}

Julia allows you to use other languages (such as Fortran or C) by using the \texttt{ccall} function:

\begin{Verbatim}
    julia> t = ccall((:clock, "libc"), Int32, ())
    2292761
\end{Verbatim}

Here, we are calling the \texttt{clock} function from the \texttt{libc} library in C.

\end{frame}

\begin{frame}[fragile]{Your legacy code}
Let's say you want to use a simply multiply function in Fortran:
\begin{Verbatim}
      !! We'll be using subroutines intead of functions 
      subroutine multiply(A, B, C)
          REAL*8 :: A, B, C
          C = A * B
          return
      end
\end{Verbatim}
\vspace{0.5cm}
or C:
\begin{Verbatim}
    // Nothing fancy here...
    double multiply(double A, double B){
        return A*B;
    }
\end{Verbatim}
\end{frame}

\begin{frame}[fragile]{Preparing your legacy code}

In order to use your favorite C or Fortran code in Julia, you need to compile it into a library, like so:

\begin{Verbatim}
    gcc -shared -O2 multiply.c -fPIC -o c_multiply.so
    gfortran -shared -O2 multiply.f90 -fPIC -o 
        fortran_multiply.so
\end{Verbatim}

\pause
These will create libraries with all of the necessary functions you could want, but beware:

\begin{center}
\textbf{C and Fortran compilers mangle function names!}
\end{center}
\end{frame}

\begin{frame}[fragile]{Using your legacy code}
There are 3 things to keep. Make sure you 
\begin{enumerate}
\item Have the right mangled name
\item Are using the right type
\item Are using the function correctly.
\end{enumerate}
\pause

\vspace{0.5cm}
For example, in C:
\begin{Verbatim}
    # This function multiplies a and b into c by using the 
    # created C library
    function call_c()
        a = Cdouble(1.0)
        b = Cdouble(3.0)
        c = ccall((:multiply, "/full/path/to/c_multiply"),
            Cdouble,(Cdouble, Cdouble),a,b)
        println(c)
    end
\end{Verbatim}
\end{frame}

\begin{frame}[fragile]{Using your legacy code}
Pointers are okay! For example, in Fortran:

\begin{Verbatim}
    # This function multiplies a and b into c by using 
    # the created FORTRAN library
    function call_fortran()
        a = Cdouble[1.0]
        b = Cdouble[2.0]
        c = Cdouble[0.0]
        ccall((:multiply_, "/full/path/to/fortran_multiply"),
           Void,(Ptr{Float64},Ptr{Float64},Ptr{Float64}),
                 a,b,c)
        println(c[1])
    end
\end{Verbatim}

\pause
More information can be found here: \url{https://docs.julialang.org/en/stable/manual/calling-c-and-fortran-code/}
\end{frame}
\end{document}
