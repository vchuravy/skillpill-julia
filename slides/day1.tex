%%%%%%%%%%%%%%%%%%%%%%%%%%%%%%%%%%%%%%%%%
% Beamer Presentation
% LaTeX Template
% Version 1.0 (10/11/12)
%
% This template has been downloaded from:
% http://www.LaTeXTemplates.com
%
% License:
% CC BY-NC-SA 3.0 (http://creativecommons.org/licenses/by-nc-sa/3.0/)
%
% Modified by Jeremie Gillet in November 2015 to make an OIST Skill Pill template
%
%%%%%%%%%%%%%%%%%%%%%%%%%%%%%%%%%%%%%%%%%

%----------------------------------------------------------------------------------------
%	PACKAGES AND THEMES
%----------------------------------------------------------------------------------------

\documentclass{beamer}

\mode<presentation> {

\usetheme{Madrid}

\definecolor{OISTcolor}{rgb}{0.65,0.16,0.16}
\usecolortheme[named=OISTcolor]{structure}

%\setbeamertemplate{footline} % To remove the footer line in all slides uncomment this line
%\setbeamertemplate{footline}[page number] % To replace the footer line in all slides with a simple slide count uncomment this line

\setbeamertemplate{navigation symbols}{} % To remove the navigation symbols from the bottom of all slides uncomment this line
}

\usepackage{graphicx} % Allows including images
\usepackage{booktabs} % Allows the use of \toprule, \midrule and \bottomrule in tables
\usepackage{textpos} % Use for positioning the Skill Pill logo
\usepackage{fancyvrb}
\usepackage{tikz}
\usepackage{hyperref}
\usepackage{listings}

\definecolor{dkgreen}{rgb}{0,0.6,0}
\definecolor{gray}{rgb}{0.5,0.5,0.5}
\definecolor{mauve}{rgb}{0.58,0,0.82}

\lstset{frame=tb,
  language=python,
  aboveskip=3mm,
  belowskip=3mm,
  showstringspaces=false,
  columns=flexible,
  basicstyle={\small\ttfamily},
  numbers=none,
  numberstyle=\tiny\color{gray},
  keywordstyle=\color{blue},
  commentstyle=\color{dkgreen},
  stringstyle=\color{mauve},
  breaklines=true,
  breakatwhitespace=true,
  tabsize=3
}

%----------------------------------------------------------------------------------------
%	TITLE PAGE
%----------------------------------------------------------------------------------------

\title[Skill Pill]{Skill Pill: Julia} % The short title appears at the bottom of every slide, the full title is only on the title page
\subtitle{Lecture 1: Introduction}

\author{James Schloss \and Valentin Churavy} % Your name
\institute[OIST] % Your institution as it will appear on the bottom of every slide, may be shorthand to save space
{
Okinawa Institute of Science and Technology \\ % Your institution for the title page
\textit{james.schloss@oist.jp}\\
\textit{valentin.churavy@oist.jp} % Your email address
}
\date{July 4, 2017} % Date, can be changed to a custom date

\begin{document}

\setbeamertemplate{background}{\includegraphics[width=\paperwidth]{SPbackground.png}} % Adding the background logo

\begin{frame}
\vspace*{1.4cm}
\titlepage % Print the title page as the first slide
\end{frame}

\setbeamertemplate{background}{} % No background logo after title frame

\addtobeamertemplate{frametitle}{}{% Adding the Skill Pill logo on the title screen after title frame
\begin{textblock*}{100mm}(.92\textwidth,-0.9cm)
\includegraphics[height=0.85cm]{julia.pdf}
\end{textblock*}}

\section{Introduction}
\begin{frame}{Installation}
  \begin{block}{Windows, Linux, and Mac OSX}
    Download a precompiled version of 0.6 from \url{https://julialang.org/downloads/}
  \end{block}
  \begin{block}{Linux and Mac OSX}
    \begin{enumerate}
      \item Use your package manager (Mac OSX: \texttt{brew cask install julia})
      \item Follow the build instructions from \url{https://github.com/JuliaLang/julia/}
    \end{enumerate}
  \end{block}
  \begin{block}{Sango and Tombo}
    OIST has Julia installed on Sango and Tombo, in case the version you need is not there let \href{mailto:it-help@oist.jp}{it-help@oist.jp} know.
  \end{block}
\end{frame}

\begin{frame}{Resources}
  \begin{description}
    \item[Documentation] \url{https://docs.julialang.org/en/release-0.6/}
    \item[Forum] \url{https://discourse.julialang.org}
    \item[Issue Tracker] \url{https://github.com/JuliaLang/julia}
    \item[Downloads] \url{https://julialang.org/downloads/}
    \item[Packages] \url{https://pkg.julialang.org/}
    \item \url{https://juliaobserver.com/}
  \end{description}
\end{frame}

\subsection{Why does Julia exist}
\begin{frame}{Why does Julia exist?}
  \pause
  \begin{block}{Hypothesis}
    \begin{enumerate}
      \item Scientist like to work in a high productivity programming language. 
      \item Eventually the problem size will increase and computational intensive
    \end{enumerate}
  \end{block}
\pause
  The old mission statement is available at \url{https://julialang.org/blog/2012/02/why-we-created-julia} and some good discussion is available at \url{https://discourse.julialang.org/t/julia-motivation-why-werent-numpy-scipy-numba-good-enough/2236} especially Stefan's comment at \url{https://discourse.julialang.org/t/julia-motivation-why-werent-numpy-scipy-numba-good-enough/2236/10}.
\pause
  \begin{block}{My personal reason}
    A fast, elegant, high level language that is fast enough to do serious numerical work on a super computer, while also having a language design that encourages efficient code.
  \end{block}

\end{frame}
\subsection{What is wrong with the status quo}
\begin{frame}{The other contenders}
  The typical languages used in science are
  \begin{enumerate}
    \item Python
    \item Matlab
    \item R
  \end{enumerate}

  Once a problem is becoming to big we usually move to
  \begin{enumerate}
    \item C/C++
    \item Fortran
  \end{enumerate}
  This is called the 2+ language problem and Julia is trying to solve that.
\end{frame}

\begin{frame}{Python and Numpy}
  \begin{itemize}
    \item Object are essentially dicts and can be changed at runtime.
    \item The compilers that exist (Numba) only work on primitive types and not user defined ones.
    \item GIL (Global Interpreter Lock) mask multi-threading hard.
    \item For fast code you need to write it in C.
    \item Numpy is great, but awful syntax for math.
  \end{itemize}
\end{frame}

\begin{frame}{Matlab}
  \begin{itemize}
    \item It costs alot of money and is not open-source.
    \item Matlab will only be fast for a subset of operations.
    \item Matlab tends to hide the computer from the programmer.
  \end{itemize}
\end{frame}

\begin{frame}{A (biased) performance comparision}
  \center
  \includegraphics[width=\linewidth]{benchmarks}
\end{frame}

\section{Getting started}
\begin{frame}[fragile]{The REPL}
  \begin{block}{The Read-Eval-Print-Loop}
    The REPL is a command-line interface to Julia and is ideal for short experiments.
    \begin{Verbatim}[fontsize=\footnotesize]
               _
   _       _ _(_)_     |  A fresh approach to technical computing
  (_)     | (_) (_)    |  Documentation: https://docs.julialang.org
   _ _   _| |_  __ _   |  Type "?help" for help.
  | | | | | | |/ _` |  |
  | | |_| | | | (_| |  |  Version 0.6.0 (2017-06-19 13:05 UTC)
 _/ |\__'_|_|_|\__'_|  |  
|__/                   |  x86_64-pc-linux-gnu

julia> 

\end{Verbatim}

In the \verb|REPL| you can use \verb|?| to switch your \verb|REPL| mode into help mode and get information about functions.
  \end{block}
\end{frame}
\begin{frame}{IDEs}
  There are two main IDEs that are \emph{feature} complete and can be used for Julia. The main one is based on Atom and is called Juno \url{http://junolab.org/}.

  The second one is based on on Visual Studio Code and available at \url{https://marketplace.visualstudio.com/items?itemName=julialang.language-julia}.

  I do not use either of them, but if that is the kind of environment you like and are used to give them a try.
\end{frame}
\begin{frame}[fragile]{Jupyter}
  Jupyter is an interactive web-based client for Python, Julia, R and many other languages.
  It offers a programming environment that is well suited for explorative data analysis or prototyping.
  \begin{block}{Installation}
  \begin{Verbatim}
    julia> ENV["JUPYTER"] = ""
    julia> Pkg.add("IJulia")
    \end{Verbatim}
  \end{block}
  \begin{block}{Starting a Jupyter session}
    \begin{Verbatim}
    julia> using IJulia
    julia> notebook()
    \end{Verbatim}
  \end{block}
  \begin{block}{JuliaBox}
    There is an online service provided by JuliaComputing at \url{https://juliabox.com} that gives you a cloud version of Jupyter.
  \end{block}
\end{frame}
\section{Language syntax and semantics}
\begin{frame}[fragile]{Variables and datatypes}
  Julia is a dynamic language and so you can simply create variables in any scope.
  \begin{lstlisting}
  x = 1   # x will be of type Int64
  y = 1.0 # y will be of type Float64
  z = 1.0 - 2.0im # z will be an Complex{Float64}
  1//2 # Rational numbers
  "This is a String"
  """
  This is a multiline
  String
  """
  'C' # Character literal
  1.0f0 # Float32 literal
  \end{lstlisting}
  Use \verb|typeof| to check the type of any variable. Variable names can be unicode and so greek symbols can be used.
  In the \verb|REPL| and most editors you can insert them by entering their \LaTeX name and press\verb|[Tab]|.
\end{frame}
\begin{frame}[fragile]{Conditionals}
  Julia has all the typical conditionals \verb|if|, \verb|else|, \verb|ifelse| which have to end in an \verb|end|.
  Blocks in Julia are not whitespace sensitive and conditionals do not need to be wrapped in round brackets.

  \begin{lstlisting}
  if rand() < 0.5
    println("Hello there!")
  else
    println("Go away!")
  end
  \end{lstlisting}

\end{frame}
\begin{frame}[fragile]{Loops}
  Julia has \verb|for| and \verb|while| loops. A \verb|while| loop takes a condition and a \verb|for| loop takes a iteratior.
  One can use \verb|break| to break out of a loop and \verb|continue| to skip to the next iteration. It is noteworthy that a \verb|for| loop can take an arbitrary iterator and even desugar tuples.
  \begin{lstlisting}
  for (i, x) in enumerate(['A', 'B', 'C'])
    if x == 'B'
      continue
    end
    println(i)
  end

  while true
    # ternary operator ?!
    rand() < 0.1 ? break : println("You are trapped!")
  end
  \end{lstlisting}
\end{frame}
\begin{frame}[fragile]{Functions and lambdas}
  Julia uses functions not scripts to organise operations. Every function is compiled for the combination of input parameters.
  \begin{lstlisting}
  """
      f(x, y)

  `f` will add two numbers together.
  """
  function f(x, y)
    return x + y
  end

  g(x) = x^2

  h = (x)-> 1/x
  map(lowercase, ['A', 'B', 'C'])
  map((x)->x+2, [1, 2, 3])
  \end{lstlisting}
\end{frame}
\begin{frame}[fragile]{Types}
  Julia's type system allows you to restrict functions to certain types and specialise functions for others.
  You can also create your own types. The names of types are typicaly captialised while functions are lowercase.
  \begin{lstlisting}
  abstract type Entity end
  mutable struct Player <: Entity 
    mass::Float64
    name::String
    position::Tuple{Float64, Float64}
  end
  struct Object <: Entity
    position::Tuple{Float64, Float64}
  end
  \end{lstlisting}
\end{frame}
\begin{frame}[fragile]{Multiple dispatch in a nutshell}
  Julia is not an Object-Oriented programming language functions do not belong to an object.
  A function is a set of multiple methods each with their own signatures. When you call a function the most specific methods is executed.

  \begin{lstlisting}
  function h(x::Number)
    println("x is most definitly a number.")
  end
  function h(x::Integer)
    println("x is a integer")
  end
  function h(x::Int8)
    println("Specific method for Int8")
  end
  \end{lstlisting}
  This becomes really powerful when having multiple argurments and being able to select the most specific method.
\end{frame}
\begin{frame}[fragile]{Modules}
  Julia code is organised as modules (namespaces). Module names are capitalised and you can nest modules as well.

  \begin{lstlisting}
  module MyModule
    export f

    g() = "Internal function"
    f() = println(g())
  end
  using MyModule
  \end{lstlisting}
\end{frame}
\section{The ecosystem}
\begin{frame}[fragile]{Installing packages}
  Julia has an inbuilt package manager called \verb|Pkg|. Julia packages end in \verb|.jl|
  and it is customary to refer to them by their full name online, but within Julia you drop the \verb|.jl|.
  So to install the Julia package \verb|Distributions.jl| in the \verb|REPL| just run:

  \begin{lstlisting}
  Pkg.add("Distributions")
  \end{lstlisting}

  A few other commands:
  \begin{lstlisting}
  # Updating the installed packages
  Pkg.update() 
  # What packages are installed?
  Pkg.status()
  \end{lstlisting}
  In order to create your own packages you have to install \verb|PkgDev.jl|.
\end{frame}
\begin{frame}{Github}
  Most of the development of Julia packages and the base language happens on Github.

  Check out \url{https://github.com/JuliaLang/julia} for the main action.
\end{frame}
\section{The foreign world, using Julia to reuse prior work}

\begin{frame}[fragile]{Using Fortran and C in Julia}

Julia allows you to use other languages (such as Fortran or C) by using the \texttt{ccall} function:

\begin{lstlisting}
    julia> t = ccall((:clock, "libc"), Int32, ())
    2292761
\end{lstlisting}

Here, we are calling the \texttt{clock} function from the \texttt{libc} library in C.

\end{frame}

\begin{frame}[fragile]{Your legacy code}
Let's say you want to use a simply multiply function in Fortran:
\begin{lstlisting}[language=fortran]
      !! We'll be using subroutines intead of functions 
      subroutine multiply(A, B, C)
          REAL*8 :: A, B, C
          C = A * B
          return
      end
\end{lstlisting}
\vspace{0.5cm}
or C:
\begin{lstlisting}[language=c]
    // Nothing fancy here...
    double multiply(double A, double B){
        return A*B;
    }
\end{lstlisting}
\end{frame}

\begin{frame}[fragile]{Preparing your legacy code}

In order to use your favorite C or Fortran code in Julia, you need to compile it into a library, like so:

\begin{lstlisting}
    gcc -shared -O2 multiply.c -fPIC -o c_multiply.so
    gfortran -shared -O2 multiply.f90 -fPIC -o 
        fortran_multiply.so
\end{lstlisting}

\pause
These will create libraries with all of the necessary functions you could want, but beware:

\begin{center}
\textbf{C and Fortran compilers mangle function names!}
\end{center}
\end{frame}

\begin{frame}[fragile]{Using your legacy code}
There are 3 things to keep. Make sure you 
\begin{enumerate}
\item Have the right mangled name
\item Are using the right type
\item Are using the function correctly.
\end{enumerate}
\pause

\vspace{0.5cm}
For example, in C:
\begin{lstlisting}
    # This function multiplies a and b into c by using the 
    # created C library
    function call_c()
        a = Cdouble(1.0)
        b = Cdouble(3.0)
        c = ccall((:multiply, "/full/path/to/c_multiply"),
            Cdouble,(Cdouble, Cdouble),a,b)
        println(c)
    end
\end{lstlisting}
\end{frame}

\begin{frame}[fragile]{Using your legacy code}
Pointers are okay! For example, in Fortran:

\begin{lstlisting}
    # This function multiplies a and b into c by using 
    # the created FORTRAN library
    function call_fortran()
        a = Cdouble[1.0]
        b = Cdouble[2.0]
        c = Cdouble[0.0]
        ccall((:multiply_, "/full/path/to/fortran_multiply"),
           Void,(Ptr{Float64},Ptr{Float64},Ptr{Float64}),
                 a,b,c)
        println(c[1])
    end
\end{lstlisting}

\pause
More information can be found here: \url{https://docs.julialang.org/en/stable/manual/calling-c-and-fortran-code/}
\end{frame}
\begin{frame}{Support for other languages}
  \begin{description}
    \item[Python] https://github.com/JuliaPy/PyCall.jl
    \item[R] https://github.com/JuliaInterop/RCall.jl
    \item[C++] https://github.com/Keno/Cxx.jl
    \item[Matlab] I have heard rumours of such a thing existing, but the horror $\cdots$
  \end{description}
  \begin{block}{Conclusion}
    Start writing Julia code now without being worried about losing your prior work!
  \end{block}
\end{frame}
\begin{frame}{What is next?}
  \begin{block}{Question?!}
    What do you want to hear learn about?
  \end{block}
  \begin{description}
    \item[Next Session] How does the compiler work and how do we get performance.
    \item[Next Tuesday] Data Structures and Algorithms
    \item[Last Session] Parallel computing, threading, GPUs? Up to grabs.
  \end{description}
\end{frame}
\end{document}
