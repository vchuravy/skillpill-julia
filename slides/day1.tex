%%%%%%%%%%%%%%%%%%%%%%%%%%%%%%%%%%%%%%%%%
% Beamer Presentation
% LaTeX Template
% Version 1.0 (10/11/12)
%
% This template has been downloaded from:
% http://www.LaTeXTemplates.com
%
% License:
% CC BY-NC-SA 3.0 (http://creativecommons.org/licenses/by-nc-sa/3.0/)
%
% Modified by Jeremie Gillet in November 2015 to make an OIST Skill Pill template
%
%%%%%%%%%%%%%%%%%%%%%%%%%%%%%%%%%%%%%%%%%

%----------------------------------------------------------------------------------------
%	PACKAGES AND THEMES
%----------------------------------------------------------------------------------------

\documentclass{beamer}

\mode<presentation> {

\usetheme{Madrid}

\definecolor{OISTcolor}{rgb}{0.65,0.16,0.16}
\usecolortheme[named=OISTcolor]{structure}

%\setbeamertemplate{footline} % To remove the footer line in all slides uncomment this line
%\setbeamertemplate{footline}[page number] % To replace the footer line in all slides with a simple slide count uncomment this line

\setbeamertemplate{navigation symbols}{} % To remove the navigation symbols from the bottom of all slides uncomment this line
}

\usepackage{graphicx} % Allows including images
\usepackage{booktabs} % Allows the use of \toprule, \midrule and \bottomrule in tables
\usepackage{textpos} % Use for positioning the Skill Pill logo
\usepackage{tikz}
\usepackage{hyperref}
\usepackage{listings}


%----------------------------------------------------------------------------------------
%	TITLE PAGE
%----------------------------------------------------------------------------------------

\title[Skill Pill]{Skill Pill: Julia} % The short title appears at the bottom of every slide, the full title is only on the title page
\subtitle{Lecture 1: Introduction}

\author{James Schloss \and Valentin Churavy} % Your name
\institute[OIST] % Your institution as it will appear on the bottom of every slide, may be shorthand to save space
{
Okinawa Institute of Science and Technology \\ % Your institution for the title page
\textit{james.schloss@oist.jp}\\
\textit{valentin.churavy@oist.jp} % Your email address
}
\date{July 4, 2017} % Date, can be changed to a custom date

\begin{document}

\setbeamertemplate{background}{\includegraphics[width=\paperwidth]{SPbackground.png}} % Adding the background logo

\begin{frame}
\vspace*{1.4cm}
\titlepage % Print the title page as the first slide
\end{frame}

\setbeamertemplate{background}{} % No background logo after title frame

\addtobeamertemplate{frametitle}{}{% Adding the Skill Pill logo on the title screen after title frame
\begin{textblock*}{100mm}(.8\textwidth,-1.25cm)
\includegraphics[height=2cm]{SPwhite.png}
\end{textblock*}}

\section{Introduction}
\begin{frame}{A short history of Julia}
  \begin{description}
    \item[dawn of time] 
    \item[0.1]
    \item[0.2]
    \item[0.3]
    \item[0.4]
    \item[0.5]
  \end{description}
\end{frame}
\begin{frame}{Installation}
  Linux: Use your package manager, download it directly or build from source
  Windows: Download
  Mac OSX: brew cask install julia, download or build from source
\end{frame}

\subsection{Why does Julia exist}
\begin{frame}{Why does Julia exist?}
  Statement: Scientist like high-level programming languages
  - Why Julia thread on discourse
  - old blogposts
\end{frame}
\subsection{What is wrong with the status quo}
\begin{frame}{The other contenders}
  The typical languages used in science are
  \begin{enumerate}
    \item Python
    \item Matlab
    \item R
  \end{enumerate}

  Once a problem is becoming to big we usually move to
  \begin{enumerate}
    \item C/C++
    \item Fortran
  \end{enumerate}
  This is called the 2+ language problem and Julia is trying to solve that.
\end{frame}

\begin{frame}{Performance vs Productivity}
  - Find that slide
\end{frame}

\begin{frame}{Python and Numpy}
  Hard to optimize, JIT limited, GIL, fast code needs to be written in C
  Numpy gets in the way when writting scientific code
\end{frame}

\begin{frame}{Matlab}
  Only fast for the subst of operation mathworks deemed important
  Cost
\end{frame}

\begin{frame}{R}
  Slow, don't even try to do numerics
\end{frame}

\begin{frame}{A (biased) performance comparision}
  Find slide
\end{frame}

\section{Getting started}
\begin{frame}{The REPL}
\end{frame}
\begin{frame}{Command line}
\end{frame}
\begin{frame}{IDEs}
  Juno/Atom
  VStudioCode
\end{frame}
\begin{frame}{Jupyter}
\end{frame}
\section{Language syntax and semantics}
\begin{frame}{Variables and datatypes}
\end{frame}
\begin{frame}{Conditionals}
\end{frame}
\begin{frame}{Loops}
\end{frame}
\begin{frame}{Functions and lambdas}
\end{frame}
\begin{frame}{Types}
\end{frame}
\begin{frame}{Multiple dispatch in a nutshell}
\end{frame}
\begin{frame}{Macros}
\end{frame}
\section{The ecosystem}
\begin{frame}{Installing packages}
\end{frame}
\section{The foreign world, using Julia to reuse prior work}
\end{document}
